% === BEGIN TEMPLATE ===

\documentclass[14pt]{extreport}

% tools
\usepackage{amsthm}
\usepackage{amssymb}
\usepackage{hyperref}
\usepackage{mathtools}
\usepackage{graphicx}
\usepackage{tikz}
\usetikzlibrary{arrows}

% theorems
\newtheorem{theorem}{Teorema}[section]
\newtheorem{corollary}{Corollario}[theorem]
\newtheorem{lemma}[theorem]{Lemma}

\theoremstyle{definition}
\newtheorem{definition}{Definizione}[section]

\theoremstyle{definition}
\newtheorem{remark}{Osservazione}[section]

% misc
\usepackage[italian]{babel} % set the language to italian
\usepackage{indentfirst} % spacing at the beginning of every paragraph
\usepackage[margin = 1.0in]{geometry} % size of the margins

\begin{document}

\begin{titlepage}
    \centering
    \vspace*{1cm}

    \textbf{\huge Progettazione di Algoritmi}

    \vspace{1.5cm}

    \textit{\Large Alessio Bandiera}

    \vfill
        
    \large Informatica, La Sapienza
\end{titlepage}

\tableofcontents

\pagebreak

% === END TEMPLATE ===

\chapter{Grafi}

\section{Grafi}

\begin{definition}[Grafo]
    Un grafo è una struttura matematica descritta da vertici, collegati da archi. Un grafo viene descritto formalmente come $G=(V, E)$, dove i $v \in V$ sono i \textit{vertici} del grafo, mentre gli $e \in E$ sono gli \textit{archi} (dall'inglese \textit{edges}). In particolare, $V(G)$ è l'insieme dei vertici di $G$, comunemente indicato con $n$, mentre $E(G)$ è l'insieme degli archi di $G$, comunemente indicato con $m$. Presi due vertici $v_1,v_2 \in V(G)$, allora $(v_1, v_2) \in E(G)$ è l'arco che li collega.
\end{definition}

\begin{remark}
    $E(G) \subseteq V^2$.
\end{remark}

\begin{definition}[Vertici adiacenti]
    $v_1, v_2 \in V(G)$ sono detti \textit{adiacenti} se $(v_1, v_2) \in E(G)$; in tal caso, si usa la notazione $v_1 \sim v_2$.
\end{definition}

\begin{definition}[Grafo indiretto]
    Un grafo è detto \textit{indiretto} se gli archi non hanno direzione, o equivalentemente $$\forall v_1, v_2 \in V(G) \quad (v_1, v_2) = (v_2, v_1) \in E(G)$$
\end{definition}

\begin{figure}[!htbp]
    \centering
    \begin{tikzpicture}[-,>=stealth',shorten >=1pt,auto,node distance=3cm,thick,main node/.style={scale=0.6,circle,draw,font=\sffamily\normalsize}]
        \node[main node] (1) {1};
        \node[main node] (2) [below left of=1] {2};
        \node[main node] (3) [below right of=2] {3};
        \node[main node] (6) [right of=1] {6};
        \node[main node] (5) [below right of=6] {5};
        \node[main node] (4) [below left of=5] {4};

        \path[every node/.style={font=\sffamily\small}]
            (1) edge (2)
            (2) edge (3)
            (2) edge (4)
            (3) edge (4)
            (4) edge (1)
            (5) edge (1)
            (6) edge (3)
            (6) edge (5)
            ;
    \end{tikzpicture}
    \caption{Un grafo indiretto.}
\end{figure}

In particolare, in questo esempio si ha che $V(G) = \{1, 2, 3, 4, 5, 6\}$ e $E(G) = \{(1, 2), (1, 4), (1, 5) (2, 3), (2, 4), (3, 4), (3, 6), (5, 6)\}$.

\begin{definition}[Grafo diretto]
    Un grafo è detto \textit{diretto} se gli archi hanno direzione, o equivalentemente $$\forall v_1, v_2 \in V(G) \quad (v_1, v_2) \neq (v_2, v_1) \in E(G)$$
\end{definition}

\begin{figure}[!htbp]
    \centering
    \begin{tikzpicture}[->,>=stealth',shorten >=1pt,auto,node distance=3cm,thick,main node/.style={scale=0.6,circle,draw,font=\sffamily\normalsize}]
        \node[main node] (1) {1};
        \node[main node] (2) [below left of=1] {2};
        \node[main node] (3) [below right of=2] {3};
        \node[main node] (6) [right of=1] {6};
        \node[main node] (5) [below right of=6] {5};
        \node[main node] (4) [below left of=5] {4};

        \path[every node/.style={font=\sffamily\small}]
            (1) edge (2)
            (2) edge (3)
            (2) edge (4)
            (3) edge (4)
            (4) edge (1)
            (5) edge (1)
            (6) edge (3)
            (6) edge (5)
            ;
    \end{tikzpicture}
    \caption{Un grafo diretto.}
\end{figure}

In particolare, in questo esempio si ha che $V(G) = \{1, 2, 3, 4, 5, 6\}$ e $E(G) = \{(1, 2), (2, 3), (2, 4), (3, 4), (4, 1), (5, 1), (6, 3), (6, 5)\}$.

\begin{definition}[Grado]
    Il \textit{grado} di un vertice $v \in V(G)$ è il numero di archi incidenti su $v$, indicato con $\deg(v)$.
\end{definition}

\begin{theorem}[Somma dei gradi]
    Dato un grafo $G$, la somma dei gradi dei vertici è pari a $2 |E(G)|$.
\end{theorem}

\begin{proof}
    Sia $G$ un grafo. Allora, ogni arco $e \in E(G)$ collega due vertici; allora necessariamente $\displaystyle \sum_{v \in V(G)}{\deg(v)} = 2 |E(G)|$.
\end{proof}

\begin{definition}[Cappio]
    Un arco con estremi coincidenti è detto \textit{cappio}.
\end{definition}

\begin{figure}[!htbp]
    \centering
    \begin{tikzpicture}[->,>=stealth',shorten >=1pt,auto,node distance=3cm,thick,main node/.style={scale=0.6,circle,draw,font=\sffamily\normalsize}]
        \node[main node] (1) {1};
        \node[main node] (2) [below left of=1] {2};
        \node[main node] (3) [below right of=2] {3};

        \path[every node/.style={font=\sffamily\small}]
            (1) edge (2)
            (2) edge (3)
            (3) edge (1)
            (3) edge [loop below] (3)
            ;
    \end{tikzpicture}
    \caption{Un grafo diretto con cappio in 3.}
\end{figure}

\begin{definition}[Grafo semplice]
    Un grafo è detto \textit{semplice} se non contiene cappi, né lati multipli, ovvero più archi per due vertici.
\end{definition}

\section{Visite}

\begin{definition}[Passeggiata]
    Una \textit{passeggiata} è una sequenza di vertici ed archi, della forma $v_0, e_1, v_1, e_2, \ldots , e_{n - 1}, v_{n - 1}, e_n, v_n$, dove $e_i=(v_{i - 1}, v_i)$. È la visita di un grafo più generale, ed è possibile ripercorrere ogni arco ed ogni vertice.
\end{definition}

\begin{remark}
    La lunghezza massima di una passeggiata su un grafo è infinita.
\end{remark}

\begin{definition}[Passeggiata chiusa]
    Una passeggiata si dice \textit{chiusa} se è della forma $v_0, e_1, v_1, e_2, \ldots , e_{n - 1}, v_{n - 1}, e_n, v_0$, dunque il primo e l'ultimo vertice coincidono. 
\end{definition}

\begin{definition}[Traccia]
    Una \textit{traccia} è una passeggiata aperta, in cui non è possibile ripercorrere gli archi, ma è possibile ripercorrere i vertici.
\end{definition}

\begin{figure}[!htbp]
    \centering
    \begin{tikzpicture}[-,>=stealth',shorten >=1pt,auto,node distance=3cm,thick,main node/.style={scale=0.6,circle,draw,font=\sffamily\normalsize}]
        \node[main node] (1) {1};
        \node[main node] (2) [below left of=1] {2};
        \node[main node] (3) [below right of=2] {3};
        \node[main node] (4) [below right of=1] {4};
        \node[main node] (5) [above right of=4] {5};
        \node[main node] (6) [below right of=4] {6};

        \path[every node/.style={font=\sffamily\small}]
            (2) edge (3)
            (2) edge (4)
            (3) edge (4)
            (4) edge (1)
            (4) edge (5)
            (4) edge (6)
            (5) edge (6)
            ;
    \end{tikzpicture}
    \caption{Un grafo indiretto.}
\end{figure}

Ad esempio, in questo grafo si ha la traccia $$\{5, (5,4), 4, (4,3), 3, (3, 2), 2, (2,4), 4, (4, 6), 6\}$$

\begin{definition}[Circuito]
    Un \textit{circuito} è una traccia chiusa.
\end{definition}

\begin{definition}[Cammino]
    Un \textit{cammino} è una traccia aperta, in cui non è possibile ripercurrere i vertici.
\end{definition}

\begin{remark}
    In una passeggiata in cui non si ripercorrono i vertici, non è possibile ripercorrere gli archi
\end{remark}

\begin{definition}[Ciclo]
    Un \textit{ciclo} è un cammino chiuso.
\end{definition}

\begin{figure}[!htbp]
    \centering
    \begin{tikzpicture}[-,>=stealth',shorten >=1pt,auto,node distance=3cm,thick,main node/.style={scale=0.6,circle,draw,font=\sffamily\normalsize}]
        \node[main node] (1) {1};
        \node[main node] (2) [below left of=1] {2};
        \node[main node] (3) [below right of=2] {3};
        \node[main node] (4) [below right of=1] {4};

        \path[every node/.style={font=\sffamily\small}]
            (1) edge (2)
            (2) edge (3)
            (2) edge (4)
            (3) edge (4)
            (4) edge (1)
            ;
    \end{tikzpicture}
    \caption{Un grafo indiretto.}
\end{figure}

In particolare, in questo esempio si hanno tre cicli: $$\{2, (2, 4), 4, (4,3), 3, (3, 2), 2\}$$ $$\{2, (2, 4), 4, (4, 1), 1, (1, 2), 2\}$$ $$\{1, (1, 2), 2, (2, 3), 3, (3, 4), 4, (4, 1), 1\}$$

\begin{definition}[Grafo connesso]
    Un grafo è detto \textit{connesso} se per ogni $v_1, v_2 \in V(G)$ esiste una passeggiata che li collega.
\end{definition}

\begin{figure}[!htbp]
    \centering
    \begin{tikzpicture}[-,>=stealth',shorten >=1pt,auto,node distance=3cm,thick,main node/.style={scale=0.6,circle,draw,font=\sffamily\normalsize}]
        \node[main node] (1) {1};
        \node[main node] (2) [below left of=1] {2};
        \node[main node] (3) [below right of=2] {3};
        \node[main node] (4) [below right of=1] {4};
        \node[main node] (5) [right of=4] {5};
        \node[main node] (6) [above right of=5] {6};
        \node[main node] (7) [below right of=5] {7};

        \path[every node/.style={font=\sffamily\small}]
            (1) edge (2)
            (2) edge (3)
            (2) edge (4)
            (3) edge (4)
            (4) edge (1)
            (5) edge (6)
            (6) edge (7)
            (7) edge (5)
            ;
    \end{tikzpicture}
    \caption{Un grafo non connesso.}
\end{figure}

In particolare, in questo esempio non esiste una passeggiata che possa collegare $4$ e $5$, dunque il grafo non è connesso.

\begin{definition}[Passeggiata euleriana]
    Una passeggiata si dice \textit{euleriana} se attraversa ogni arco del grafo, senza ripercorrerne nessuno.
\end{definition}

\begin{remark}[Passeggiata euleriana]
    Una passeggiata euleriana è una traccia passante per ogni arco del grafo.
\end{remark}

\begin{figure}[!htbp]
    \centering
    \begin{tikzpicture}[-,>=stealth',shorten >=1pt,auto,node distance=3cm,thick,main node/.style={scale=0.6,circle,draw,font=\sffamily\normalsize}]
        \node[main node] (1) {1};
        \node[main node] (2) [below left of=1] {2};
        \node[main node] (3) [below right of=2] {3};

        \path[every node/.style={font=\sffamily\small}]
            (1) edge (2)
            (2) edge (3)
            ;
    \end{tikzpicture}
    \caption{Un grafo indiretto.}
\end{figure}

In particolare, in questo esempio si ha la passeggiata euleriana: $$\{1, (1,2), 2, (2,3), 3\}$$

\begin{theorem}
    Dato un grafo $G$, esiste un circuito euleriano su $G$ se e solo se $G$ è connesso, e per ogni $v$, $\deg(v)$ è pari.
\end{theorem}

\begin{proof}
    \textit{Prima implicazione.} Sia $G$ un grafo avente un circuito euleriano; per assurdo, sia $v \in V(G) \mid \deg(v)$ non sia pari. Allora, percorrendo $G$ secondo il circuito euleriano, giungendo a $v$ non si potrebbe più lasciare tale vertice senza riattraversare uno degli archi gia visitati. Inoltre, se $G$ non fosse connesso, il circuito non potrebbe essere euleriano poiché non potrebbe attraversare tutti gli archi di $G$. \textit{Seconda implicazione.} TODO
\end{proof}

\begin{definition}[Passeggiata hamiltoniana]
    Una passeggiata si dice \textit{hamiltoniana} se TODO
\end{definition}

\section{Rappresentazione}

\begin{definition}[Matrice di adiacenza]
    Sia $G = (V, E)$ un grafo; allora, è possibile rappresentare $G$ attraverso una matrice $\mathcal{G} \in \textrm{Mat}_{n \times n}(\{0, 1\})$, dove $$\forall g_{i, j} \in \mathcal{G} \quad g_{i, j} = \left \{ \begin{array}{ll} 1 & i \sim j\\ 0 & i \nsim j \end{array} \right.$$
\end{definition}

\end{document}
