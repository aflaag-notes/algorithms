% === BEGIN TEMPLATE ===

\documentclass[14pt]{extreport}

% tools
\usepackage{amsthm}
\usepackage{hyperref}
\usepackage{mathtools}
\usepackage{graphicx}
\usepackage{tikz}
\usetikzlibrary{arrows}

% theorems
\newtheorem{theorem}{Teorema}[section]
\newtheorem{corollary}{Corollary}[theorem]
\newtheorem{lemma}[theorem]{Lemma}

\theoremstyle{definition}
\newtheorem{definition}{Definizione}[section]

\theoremstyle{remark}
\newtheorem*{remark}{Remark}

% misc
\usepackage[italian]{babel} % set the language to italian
\usepackage{indentfirst} % spacing at the beginning of every paragraph
\usepackage[margin = 1.0in]{geometry} % size of the margins

\begin{document}

\begin{titlepage}
    \centering
    \vspace*{1cm}

    \textbf{\huge Algoritmi II}

    \vspace{1.5cm}

    \textit{\Large Alessio Bandiera}

    \vfill
        
    \large Informatica, La Sapienza
\end{titlepage}

\tableofcontents

\pagebreak

% === END TEMPLATE ===

\chapter{Grafi}

\section{Grafi}

\begin{definition}[Grafo]
\
    Un grafo è una struttura matematica descritta da vertici, collegati da archi. Un grafo viene descritto formalmente come $G=(V, E)$, dove $V$ sono i \textbf{vertici} del grafo, mentre $E$ sono gli \textbf{archi} (dall'inglese \textit{edges}). In particolare, $V(G)$ è l'insieme dei vertici di $G$, comunemente indicato con $n$, mentre $E(G)$ è l'insieme degli archi di $G$, comunemente indicato con $m$.
\end{definition}

\begin{figure}[h]
    \centering
    \begin{tikzpicture}
        [-,>=stealth',shorten >=1pt,auto,node distance=3cm,thick,main node/.style={circle,draw,font=\sffamily\normalsize}]

        \node[main node] (1) {1};
        \node[main node] (2) [below left of=1] {2};
        \node[main node] (3) [below right of=2] {3};
        \node[main node] (6) [right of=1] {6};
        \node[main node] (5) [below right of=6] {5};
        \node[main node] (4) [below left of=5] {4};

        \path[every node/.style={font=\sffamily\small}]
        (1) edge (2)
        (2) edge (3)
        (2) edge (4)
        (3) edge (4)
        (4) edge (1)
        (5) edge (1)
        (6) edge (3)
        (6) edge (5)
        ;
    \end{tikzpicture}
    \caption{Un grafo}
\end{figure}

\begin{lemma}[Cammini euleriani]
    Cacca
\end{lemma}

\begin{proof}
\end{proof}

\pagebreak

\end{document}
